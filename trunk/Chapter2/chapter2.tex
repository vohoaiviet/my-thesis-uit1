\chapter{Các công trình liên quan}
\ifpdf
    \graphicspath{{Chapter2/Chapter2Figs/PNG/}{Chapter2/Chapter2Figs/PDF/}{Chapter2/Chapter2Figs/}}
\else
    \graphicspath{{Chapter2/Chapter2Figs/EPS/}{Chapter2/Chapter2Figs/}}
\fi

Trong chương này chúng tôi sẽ trình bày một cách tổng quan về các phương pháp truy vấn đối tượng trên tập dữ liệu ảnh lớn đang được sử dụng rộng rãi hiện nay. Các phương pháp cần phải thỏa hai yêu cầu là cho kết quả với độ chính xác cao và trả về trong thời gian gần như ngay lập tức.\\
Để có thể truy vấn hình ảnh trong thời gian ngắn, mọi dữ liệu phải được lưu trữ trên RAM vì tốc độ truy xuất ổ cứng rất chậm. Tuy nhiên do dung lượng rất hạn chế của RAM, ta phải tìm cách biểu diễn tập dữ liệu hình ảnh cho phù hợp để vừa đảm bảo được về mặt không gian lưu trữ, vừa đáp ứng được các yêu cầu của truy vấn ảnh. Mục \ref{local-features} sẽ trình bày ngắn gọn về hướng tiếp cận biểu diễn hình ảnh bằng các đặc trưng cục bộ. Nhưng khi kích cỡ của tập dữ liệu tăng thì việc so khớp các đặc trưng cục bộ tỏ ra kém hiệu quả. Trong mục \ref{bag-of-words}, chúng tôi sẽ giới thiệu mô hình Bag-of-visual-words -  được bắt nguồn từ mô hình Bag-of-Words (BoW) trong truy vấn văn bản. Mô hình này cho thấy tính hiệu quả của nó cả về tốc độ tính toán lẫn bộ nhớ sử dụng.\\
Mặc dù đạt được hiệu suất cao nhưng mô hình BoW vẫn bỏ qua thông tin về không gian ảnh - một thông tin quan trọng ảnh hướng lớn đến độ chính xác của truy vấn. Trong mục \ref{spatial}, chúng tôi sẽ trình bày rõ hơn về các hướng tiếp cận dựa để khai thác được thông tin không gian ảnh, tiêu biểu là phương pháp RASAC và Spatial Pyramid Matching (SPM).\\



\section{Biểu diễn hình ảnh bằng các đặc trưng cục bộ}
\label{local-features}

Trong lĩnh vực Thị giác Máy tính, một câu hỏi và cũng là một thách thức lớn đối với tất cả các nhà khoa học là làm sao biểu diễn được môt hình ảnh trên máy tính. Tùy theo từng mục đích cụ thể, người ta sẽ có các cách biểu diễn khác nhau. Trong truy vấn ảnh, một hình ảnh phải được biểu diễn dưới dạng sao cho bền vững trước những thay đổi như điều kiện chụp, tỉ lệ, góc chụp khác nhau hay thậm chí là những thay đổi lớn do đối tượng bị che khuất. Do sự tác động của các yếu tố này, cho dù hai hình ảnh chứa cùng một đối tượng thì vẫn có thể tồn tại một vùng hình ảnh lớn bên ngoài các đối tượng không đồng thời xuất hiện ở cả hai hình.\\
Để giải quyết vấn đề này, có một hướng tiếp cận phổ biến là rút trích những "chi tiết" cục bộ (local patches) trên tấm hình để biểu diễn cho hình ảnh đó. Hướng tiếp cận này được đưa ra dựa trên nhận định rằng hai hình ảnh tương tự nhau sẽ có rất nhiều những chi tiết cục bộ giống nhau và những chi tiết cục bộ này có thể được dùng để so khớp các hình ảnh với nhau. Các chi tiết này thường được rút trích bằng một trong hai phương pháp, đó là: (i) sử dụng một lưới dày đặc với nhiều mức tỉ lệ kích cỡ khác nhau (để đảm bảo bất biết về tỉ lệ) để chia hình ảnh thành nhiều chi tiết nhỏ, hoặc (ii) dùng các phương pháp dò tìm (detector) hay một kỹ thuật nào đó để lấy được các chi tiết đặc biệt (đặc trưng) trên vùng hình ảnh quan tâm và đồng thời loại bỏ những chi tiết không đảm bảo sự bất biến tỉ lệ ngay ở bước này. Có thể thấy rằng phương pháp dùng lưới để chia hình ảnh thành nhiều phần không thể áp dụng cho bài toán truy vấn ảnh với tập dữ liệu lớn vì ta cần rất nhiều không gian để lưu trữ một lượng lớn các chi tiết dày đặc với nhiều mức tỉ lệ kích cỡ khác nhau. Do vậy phương pháp biểu diễn hình ảnh bằng các đặc trưng được áp dụng cho bài toán này.\\
Có rất nhiều phương pháp dò tìm các đặc trưng (feature detector) được đưa ra, trong đó phải kể tới các phương pháp được dùng phổ biến như Difference of Gaussians, DoG (\cite{lowe2004distinctive}), Maximally Stable Extremal Regions, MSER (\cite{matas2004robust}) và affine invariant detector (\cite{mikolajczyk2004scale}). Ngoài ra còn có các phương pháp dò tìm được xây dựng để tìm kiếm trong thời gian thực như SURF (\cite{bay2006surf}), FAST (\cite{rosten2010faster}) và BRISK (\cite{leutenegger2011brisk}).\\
Sau khi rút trích được các đặc trưng cục bộ cho mỗi hình, dựa trên các đặc trưng đó ta sẽ quyết định xem liệu hai tấm hình bất kỳ có chứa cùng một đối tượng hay không. Để so sánh độ tương đồng của hai đặc trưng cục bộ, ta không thể dựa trên màu sắc và cường độ của chúng vì những yếu tố này không bền vững trước những thay đổi của hình ảnh. Do đó ta cần phải tìm cách lượng tử hóa độ tương đồng giữa cách đặc trưng để có thể đo được bằng các tính toán cụ thể. Trong công trình nghiên cứu nổi tiếng của \cite{lowe2004distinctive}, tác giả đã đề xuất một phương pháp để có thể tính toán được một bộ mô tả (descriptor) có tính phân loại cao và đảm bảo sự bất biến trước những thay đổi của hình ảnh, đó là SIFT descriptor. Theo sau công trình nghiên cứu này, nhiều công trình có hướng tiếp cận tương tự được đưa ra, trong đó bao gồm GLOH (\cite{mikolajczyk2005performance}), SURF (\cite{bay2006surf}), DAISY (\cite{tola2008fast}), CONGAS (\cite{zheng2009tour}), BRIEF (\cite{calonder2010brief}). Đặc biệt, bằng việc đề xuất thuật toán RootSIFT được cải tiến từ SIFT, \cite{arandjelovic2012three} đã nâng hiệu suất của phương pháp SIFT lên đáng kể. Đây cũng là phương pháp được chúng tôi chọn dùng trong hệ thống của mình.\\
Tóm lại, từ những bộ mô tả (descriptor) được rút trích từ tất cả các hình trong cơ sở dữ liệu và từ hình ảnh truy vấn, ta có thể tính toán được độ tương đồng giữa các hình ảnh. Tuy nhiên, hiệu suất của quá trình tính toán độ tương đồng bị giảm đi đáng kể khi thực hiện trên tập dữ liệu lớn. Trong phần tiếp theo, chúng tôi sẽ giới thiệu sơ lược về một mô hình giúp giải quyết được vấn đề này.

\section{Mô hình Bag-of-words}
\label{bag-of-words}
Mô hình bag-of-words đã thể hiện được sức mạnh của nó trong truy vấn văn bản và được sử dụng trong các công cụ tìm kiếm văn bản mạnh mẽ như Google, Bing. Chính vì sự thành công đó, bag-of-words đã được sử dụng trong truy vấn ảnh. Mục này chủ yếu trình bày về việc ứng dụng phương pháp truy vấn văn bản này vào trong truy vấn ảnh. Trước tiên, chúng tôi sẽ sơ lược về truy vấn văn bản, tiếp đến sẽ là việc ứng dụng của nó trong truy vấn ảnh.

\subsection{Bag-of-words trong truy vấn văn bản}
Tương tự như hình ảnh, để có thể thực hiện truy vấn với văn bản, văn bản được biểu diễn dưới dạng một mô hình không gian vector (\cite{Salton:1986:IMI:576628}) hay còn được gọi là mô hình \textit{túi từ} (bag-of-words), BoW (\cite{manning2008introduction}). Theo đó, mỗi văn bản được xem như là một tập hỗn độn (một túi) các từ và được biểu diễn dưới dạng một biểu đồ (histogram) \textit{N\textsubscript{w}}-chiều với \textit{N\textsubscript{w}} là số các từ của một ngôn ngữ. Vì giá trị của mỗi cột cột của biểu đồ bằng với số lần xuất hiện của từ tương ứng với cột đó trong văn bản nên phương pháp này còn được gọi là \textit{trọng số tần suất từ} (term frequency weighting).\\
Đôi khi, chúng ta có thể bắt gặp trường hợp nhiều từ xuất hiện trong các văn bản nhiều hơn các từ khác (ví dụ như trong tiếng Anh là \textit{the} và \textit{and}). Tuy nhiên những từ này thường mang ít giá trị hơn những từ ít phổ biến trong việc phục vụ cho mục đích so khớp. Do sự mất cân đối trong tần số xuất hiện của các từ, các chiều trong mô hình không gian vector phải được đánh trọng số dựa trên giá trị của thông tin mà từ đó mang chứ không phải dựa trên tần suất xuất hiện. Một phương pháp đánh trọng số thường được sử dụng là \textit{tần số văn bản nghịch đảo}, idf (invert document frequency). Với \textit{N\textsubscript{D}} là tổng số các văn bản, \textit{N\textsubscript{i}} là số văn bản mà từ \textit{i} xuất hiện, công thức tính \textit{tần số văn bản nghịch đảo} được phát biểu như sau:
\begin{eqnarray}
 idf\textsubscript{i} &=& log \frac{N\textsubscript{D}}{N\textsubscript{i}}
\end{eqnarray}
Cuối cùng, trọng số của mỗi từ trong mỗi văn bản được tính bằng cách lấy tích của tần suất từ (term frequency - tf) và tần số nghịch đảo văn bản (invert document frequency - idf). Trọng số đó được gọi là tf-idf (\cite{manning2008introduction}) với công thức:
\begin{eqnarray}
 tf-idf\textsubscript{i,d} &=& tf\textsubscript{i,d} \times idf\textsubscript{i}
\end{eqnarray}
Đối với những từ xuất hiện với tần suất cực kỳ lớn (stop word), ta có thể lọc và loại bỏ toàn bộ để giảm bớt chi phí về không gian lưu trữ và thời gian thực thi.\\
Mức độ tương đồng giữa các văn bản sẽ được tính bằng công thức cosin áp dụng cho trọng số tf-idf của chúng trong mô hình bag-of-words. Thực tế mỗi văn bản chỉ chứa một lượng rất nhỏ so với số lượng các từ có trong ngôn ngữ, do vậy vector sinh ra khi biểu diễn bằng mô hình bag-of-words sẽ rất thưa thớt. Để cho quá trình lưu trữ và truy vấn được hiệu quả, một cấu trúc dữ liệu sẽ được tính toán trước được gọi là \textit{chỉ mục ngược} (inverted index). Chỉ mục ngược bao gồm một chuỗi các danh sách, mỗi danh sách tương ứng với một từ. Mỗi danh sách ghi lại những văn bản nào có chứa từ đó. Nhờ chỉ mục ngược, khi đưa vào một danh sách các từ truy vấn rút từ văn bản truy vấn, ta có thể nhanh chóng lấy được danh sách các văn bản trong tập văn bản chứa các từ truy vấn đó. Từ đó có thể dễ dàng tính ra chỉ số tf-idf cho từng từ.

\subsection{Bag-of-visual-words trong truy vấn ảnh}
Lorem ipsum dolor sit amet, consectetur adipiscing elit. Sed porta nisl lectus, condimentum tristique eros auctor eu. Duis iaculis metus urna, posuere lacinia mi rhoncus id. In molestie augue quis semper scelerisque. Sed egestas nulla ac velit consectetur tristique. Nulla luctus purus tortor, ut mattis mauris tincidunt quis. Proin pulvinar nisl vitae dui varius, sit amet aliquam ipsum mollis. Aliquam erat volutpat. Sed laoreet justo in orci venenatis, quis rhoncus sapien tincidunt. Curabitur bibendum scelerisque erat, ac fringilla justo.\\
   \subsubsection{Visual word}
   
   Phần này sẽ trình bày sơ lược về visual words và các nghiên cứu liên quan tới visual words (các thuật toán cluster k-means, AKM, HKM).\\
Lorem ipsum dolor sit amet, consectetur adipiscing elit. Nullam leo purus, condimentum vel massa sit amet, suscipit congue leo. Etiam et porttitor tortor. In eget leo orci. Vivamus faucibus eget justo in sodales. Etiam eu tempor diam. Morbi a auctor massa, non cursus purus. Nullam elementum luctus gravida. Praesent tristique mi sit amet nulla pharetra viverra.\\

In eget diam euismod, lacinia nisi ut, lacinia orci. Donec varius felis vel eleifend sodales. Integer neque arcu, dapibus ut laoreet id, elementum iaculis felis. Proin suscipit varius velit, sit amet tempor mauris auctor sed. Curabitur ultricies consectetur dignissim. Sed interdum felis in magna sagittis, id iaculis nibh accumsan. Aliquam lacus purus, auctor tincidunt felis sit amet, mattis vulputate nunc. Cras dictum, erat sed aliquam molestie, metus magna tempus felis, interdum mollis est turpis commodo tellus. Interdum et malesuada fames ac ante ipsum primis in faucibus. Quisque dapibus quam nec vulputate malesuada. Fusce lectus dui, ullamcorper sit amet ullamcorper ac, adipiscing in enim. Proin ipsum felis, tristique in mi vel, suscipit ullamcorper justo. Sed lobortis eros vitae tortor luctus commodo.\\
   \subsubsection{Lượng tử hóa các visual word}
   Phần này trình bày về việc lượng tử hóa các visual word và việc đo độ tương đồng giữa các visual word. \\   
Lorem ipsum dolor sit amet, consectetur adipiscing elit. Nullam leo purus, condimentum vel massa sit amet, suscipit congue leo. Etiam et porttitor tortor. In eget leo orci. Vivamus faucibus eget justo in sodales. Etiam eu tempor diam. Morbi a auctor massa, non cursus purus. Nullam elementum luctus gravida. Praesent tristique mi sit amet nulla pharetra viverra.\\

In eget diam euismod, lacinia nisi ut, lacinia orci. Donec varius felis vel eleifend sodales. Integer neque arcu, dapibus ut laoreet id, elementum iaculis felis. Proin suscipit varius velit, sit amet tempor mauris auctor sed. Curabitur ultricies consectetur dignissim. Sed interdum felis in magna sagittis, id iaculis nibh accumsan. Aliquam lacus purus, auctor tincidunt felis sit amet, mattis vulputate nunc. Cras dictum, erat sed aliquam molestie, metus magna tempus felis, interdum mollis est turpis commodo tellus. Interdum et malesuada fames ac ante ipsum primis in faucibus. Quisque dapibus quam nec vulputate malesuada. Fusce lectus dui, ullamcorper sit amet ullamcorper ac, adipiscing in enim. Proin ipsum felis, tristique in mi vel, suscipit ullamcorper justo. Sed lobortis eros vitae tortor luctus commodo.\\
\subsection{Đánh giá sự tương đồng giữa các hình ảnh}

Lorem ipsum dolor sit amet, consectetur adipiscing elit. Nullam leo purus, condimentum vel massa sit amet, suscipit congue leo. Etiam et porttitor tortor. In eget leo orci. Vivamus faucibus eget justo in sodales. Etiam eu tempor diam. Morbi a auctor massa, non cursus purus. Nullam elementum luctus gravida. Praesent tristique mi sit amet nulla pharetra viverra.\\

In eget diam euismod, lacinia nisi ut, lacinia orci. Donec varius felis vel eleifend sodales. Integer neque arcu, dapibus ut laoreet id, elementum iaculis felis. Proin suscipit varius velit, sit amet tempor mauris auctor sed. Curabitur ultricies consectetur dignissim. Sed interdum felis in magna sagittis, id iaculis nibh accumsan. Aliquam lacus purus, auctor tincidunt felis sit amet, mattis vulputate nunc. Cras dictum, erat sed aliquam molestie, metus magna tempus felis, interdum mollis est turpis commodo tellus. Interdum et malesuada fames ac ante ipsum primis in faucibus. Quisque dapibus quam nec vulputate malesuada. Fusce lectus dui, ullamcorper sit amet ullamcorper ac, adipiscing in enim. Proin ipsum felis, tristique in mi vel, suscipit ullamcorper justo. Sed lobortis eros vitae tortor luctus commodo.\\
\section{Sử dụng thông tin không gian ảnh trong truy vấn ảnh}
\label{spatial}
Lorem ipsum dolor sit amet, consectetur adipiscing elit. Mauris sit amet sollicitudin nibh, sodales rutrum metus. Morbi faucibus tincidunt felis, at pretium tellus placerat ut. Phasellus aliquet leo a vestibulum eleifend. Phasellus fermentum eros ut erat fringilla posuere. Suspendisse sodales dictum turpis, et pulvinar magna sodales rutrum. Quisque vel commodo sem. Vivamus arcu risus, commodo eu facilisis sed, posuere vel nisi. Proin risus ipsum, hendrerit vitae venenatis ut, fringilla quis nunc.

Ut eget erat vitae odio gravida adipiscing. Pellentesque rutrum sit amet odio eget ornare. Sed non blandit ligula. Mauris mollis nibh sed eros consectetur, at semper erat dictum. Nam dolor purus, convallis a mauris aliquet, consectetur rhoncus leo. Cras nec erat eleifend, cursus purus mattis, dapibus lacus. Duis tortor nibh, consectetur eget arcu vitae, mollis laoreet magna. Vestibulum tincidunt augue ut condimentum consequat. Vestibulum vel neque mauris. Interdum et malesuada fames ac ante ipsum primis in faucibus. Nullam risus est, posuere in rhoncus a, varius non elit. Fusce a dictum eros, eget vestibulum tellus. Nunc at aliquet velit. Fusce eget quam mi. Vestibulum tristique, ligula id condimentum dapibus, nunc nisi consequat magna, vel luctus lorem quam eget velit.

Sed egestas eu lorem sollicitudin blandit. Morbi et iaculis nisi, ac consequat sem. Duis elementum leo sit amet dolor eleifend, id gravida dui mattis. Praesent malesuada, eros ac consequat auctor, ante justo lobortis dui, id malesuada lectus dolor at lorem. Phasellus ultricies rhoncus semper. Etiam lacus ligula, vehicula ut metus ac, sagittis varius quam. Nulla ut pretium tortor. Nunc feugiat, arcu et posuere consectetur, nibh diam porta eros, a consectetur leo ante ac ante. Integer malesuada laoreet augue, quis scelerisque tortor varius non. Phasellus a quam ut est tempus pharetra. Duis hendrerit eros ut lacinia tincidunt. Aenean non molestie neque. Mauris dapibus non massa eget ullamcorper. Quisque id urna vestibulum neque feugiat egestas. Praesent at egestas massa.

\section{Kết chương}
Lorem ipsum dolor sit amet, consectetur adipiscing elit. Vestibulum lectus massa, pellentesque ac rutrum sed, venenatis posuere magna. Aenean eget sagittis sem. Fusce a tortor in odio congue vehicula at condimentum urna. Suspendisse nec tortor nec tortor rutrum laoreet vitae nec lacus. Quisque sed purus vitae ipsum porta malesuada. Nullam sed turpis vitae nulla fringilla ullamcorper. Aenean vitae ligula enim. Suspendisse eu molestie nulla. Praesent id lacus tincidunt, lacinia eros sit amet, sagittis urna. Etiam nisl purus, varius non quam vel, pretium tincidunt est. Aliquam erat volutpat. Fusce auctor mattis neque, ut ultrices nisl gravida at. Etiam vel placerat erat. Donec rutrum viverra lacus sit amet gravida. Fusce vel nibh a risus condimentum feugiat eget vitae enim. Curabitur sem sem, convallis vitae consectetur id, condimentum sit amet libero.

