\begin{abstracts}         
Trong những năm gần đây, truy vấn ảnh trên tập dữ liệu lớn là bài toán đang thu hút được nhiều sự quan tâm và có ý nghĩa quan trọng trong thực tiễn. Bài toán trên có thể phát biểu như sau: Đưa vào một hình ảnh có chứa đối tượng quan tâm và ngay lập tức trả về những hình ảnh có chứa đối tượng đó từ một tập dữ liệu trong thời gian thực. Các hệ thống truy vấn ảnh trên cơ sở dữ liệu lớn có nhiều ứng dụng quan trọng trong các lĩnh vực như nhận dạng đối tượng hay địa điểm, tìm kiếm video, phát hiện trùng lặp và tái tạo 3D, v.v... Tuy nhiên, bài toán trên cũng đang đối mặt với nhiều thách thức. Bên cạnh vấn đề về sự xuất hiện các biến thể của hình ảnh của đối tượng do sự khác nhau về độ sáng, kích thước, góc chụp hay bị che khuất một phần thì ở đây còn một vấn đề quan trọng khác là phải đảm bảo được thời gian thực hiện truy vấn đặc biệt là khi tìm kiếm trong tập dữ liệu lớn.

Rất nhiều công trình nghiên cứu đã được đề xuất để giải quyết vấn đề trên và đã đạt được nhiều bước tiến đáng chú ý. Hầu hết các công trình này đều dựa trên mô hình Bag-of-Words (BoW), theo đó mỗi hình ảnh sẽ được biểu diễn bằng các đặc trưng cục bộ, sau đó các đặc trưng này được lượng tử hóa vào các visual word. Để tăng hiệu suất của quá trình truy vấn, người ta thường sử dụng mô hình Bag-of-Words kết hợp với phương pháp đánh chỉ mục ngược (Inverted Index). Thế nhưng cả Bag-of-Words và Inverted Index đều bỏ qua một thông tin quan trọng để tăng độ chính xác cho truy vấn, đó là thông tin không gian ảnh (spatial information) của các đặc trưng cục bộ. 

Trong luận văn này, chúng tôi đề xuất một phương pháp nhằm tích hợp thông tin không gian ảnh vào phương pháp đánh chỉ mục ngược (Inverted Index) để nâng cao độ chính xác nhưng vẫn đảm bảo được thời gian truy vấn nhanh. Kết quả thí nghiệm trên các tập dữ liệu chuẩn như Oxford 5k, Paris 6k và Holiday đã cho thấy tính hiệu quả của phương pháp này.\\
\\
\textit{Từ khóa: Tìm kiếm ảnh - Image Search, Kích cỡ lớn - Large-Scale, Thông tin không gian - Spatial Information, Chỉ mục ngược - Inverted Index.}

\end{abstracts}
 