\begin{abstracts}         
Trong những năm gần đây, truy vấn ảnh trên tập dữ liệu lớn là bài toán đang thu hút được nhiều sự quan tâm và có ý nghĩa quan trọng trong thực tiễn. Bài toán trên có thể phát biểu như sau: Đưa vào một hình ảnh có chứa đối tượng quan tâm và ngay lập tức trả về những hình ảnh có chứa đối tượng đó từ một tập dữ liệu trong thời gian thực. Các hệ thống truy vấn ảnh trên cơ sở dữ liệu lớn có nhiều ứng dụng quan trọng trong các lĩnh vực như nhận dạng đối tượng hay địa điểm, tìm kiếm video, phát hiện trùng lặp và tái tạo 3D, v.v... Tuy nhiên, bài toán trên cũng đang đối mặt với nhiều thách thức. Bên cạnh vấn đề về sự xuất hiện các biến thể của hình ảnh của đối tượng do sự khác nhau về độ sáng, kích thước, góc chụp hay bị che khuất một phần thì ở đây còn một vấn đề quan trọng khác là phải đảm bảo được thời gian thực hiện truy vấn đặc biệt là khi tìm kiếm trong tập dữ liệu lớn.\\
Rất nhiều công trình nghiên cứu đã được đề xuất để giải quyết vấn đề trên và đã đạt được nhiều bước tiến lớn. Hầu hết các công trình đó đều dựa trên mô hình Bag-of-Words (BoW), tức là mỗi hình ảnh sẽ được biểu diễn bằng các đặc trưng cục bộ, sau đó các đặc trưng này được lượng tử hóa vào các visual word. Và phương pháp đánh chỉ mục ngược (Inverted Index) thường được sử dụng để tăng hiệu suất của quá trình truy vấn.


\end{abstracts}
 